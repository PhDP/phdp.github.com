%%%%%%%%%%%%%%%%%%%%%%%%%%%%%%%%%%%%%%%%%%%%%%%%%%%%%%%%%%%%%%%%%%%%%%%%
%%%%%%%%%%%%%%%%%%%%%% Simple LaTeX CV Template %%%%%%%%%%%%%%%%%%%%%%%%
%%%%%%%%%%%%%%%%%%%%%%%%%%%%%%%%%%%%%%%%%%%%%%%%%%%%%%%%%%%%%%%%%%%%%%%%

%%%%%%%%%%%%%%%%%%%%%%%%%%%%%%%%%%%%%%%%%%%%%%%%%%%%%%%%%%%%%%%%%%%%%%%%
%% NOTE: If you find that it says                                     %%
%%                                                                    %%
%%                           1 of ??                                  %%
%%                                                                    %%
%% at the bottom of your first page, this means that the AUX file     %%
%% was not available when you ran LaTeX on this source. Simply RERUN  %%
%% LaTeX to get the ``??'' replaced with the number of the last page  %%
%% of the document. The AUX file will be generated on the first run   %%
%% of LaTeX and used on the second run to fill in all of the          %%
%% references.                                                        %%
%%%%%%%%%%%%%%%%%%%%%%%%%%%%%%%%%%%%%%%%%%%%%%%%%%%%%%%%%%%%%%%%%%%%%%%%

%%%%%%%%%%%%%%%%%%%%%%%%%%%% Document Setup %%%%%%%%%%%%%%%%%%%%%%%%%%%%

% Don't like 10pt? Try 11pt or 12pt
\documentclass[10pt]{article}
\usepackage[utf8]{inputenc}

% The automated optical recognition software used to digitize resume
% information works best with fonts that do not have serifs. This
% command uses a sans serif font throughout. Uncomment both lines (or at
% least the second) to restore a Roman font (i.e., a font with serifs).
%\usepackage{times}
%\renewcommand{\familydefault}{\sfdefault}

% This is a helpful package that puts math inside length specifications
\usepackage{calc}
\usepackage[usenames,dvipsnames]{xcolor}

\def\us{\char`\_}

% Layout: Puts the section titles on left side of page
\reversemarginpar

%
%         PAPER SIZE, PAGE NUMBER, AND DOCUMENT LAYOUT NOTES:
%
% The next \usepackage line changes the layout for CV style section
% headings as marginal notes. It also sets up the paper size as either
% letter or A4. By default, letter was used. If A4 paper is desired,
% comment out the letterpaper lines and uncomment the a4paper lines.
%
% As you can see, the margin widths and section title widths can be
% easily adjusted.
%
% ALSO: Notice that the includefoot option can be commented OUT in order
% to put the PAGE NUMBER *IN* the bottom margin. This will make the
% effective text area larger.
%
% IF YOU WISH TO REMOVE THE ``of LASTPAGE'' next to each page number,
% see the note about the +LP and -LP lines below. Comment out the +LP
% and uncomment the -LP.
%
% IF YOU WISH TO REMOVE PAGE NUMBERS, be sure that the includefoot line
% is uncommented and ALSO uncomment the \pagestyle{empty} a few lines
% below.
%

%% Use these lines for letter-sized paper
\usepackage[paper=letterpaper,
            %includefoot, % Uncomment to put page number above margin
            marginparwidth=1.2in,     % Length of section titles
            marginparsep=.05in,       % Space between titles and text
            margin=1in,               % 1 inch margins
            includemp]{geometry}

%% Use these lines for A4-sized paper
%\usepackage[paper=a4paper,
%            %includefoot, % Uncomment to put page number above margin
%            marginparwidth=30.5mm,    % Length of section titles
%            marginparsep=1.5mm,       % Space between titles and text
%            margin=25mm,              % 25mm margins
%            includemp]{geometry}

%% More layout: Get rid of indenting throughout entire document
\setlength{\parindent}{0in}

\usepackage[shortlabels]{enumitem}

% Simpler bibsections for CV sections
% (thanks to natbib for inspiration)
%
% * For lists of references with hanging indents and no numbers:
%
%   \begin{bibsection}
%       \item ...
%   \end{bibsection}
%
% * For numbered lists of references (with hanging indents):
%
%   \begin{bibenum}
%       \item ...
%   \end{bibenum}
%
%   Note that bibenum numbers continuously throughout. To reset the
%   counter, use
%
%   \restartlist{bibenum}
%
%   at the place where you want the numbering to reset.

\makeatletter
\newlength{\bibhang}
\setlength{\bibhang}{1em}
\newlength{\bibsep}
 {\@listi \global\bibsep\itemsep \global\advance\bibsep by\parsep}
\newlist{bibsection}{itemize}{3}
\setlist[bibsection]{label=,leftmargin=\bibhang,%
        itemindent=-\bibhang,
        itemsep=\bibsep,parsep=\z@,partopsep=0pt,
        topsep=0pt}
\newlist{bibenum}{enumerate}{3}
\setlist[bibenum]{label=[\arabic*],resume,leftmargin={\bibhang+\widthof{[999]}},%
        itemindent=-\bibhang,
        itemsep=\bibsep,parsep=\z@,partopsep=0pt,
        topsep=0pt}
\let\oldendbibenum\endbibenum
\def\endbibenum{\oldendbibenum\vspace{-.6\baselineskip}}
\let\oldendbibsection\endbibsection
\def\endbibsection{\oldendbibsection\vspace{-.6\baselineskip}}
\makeatother

%% Reference the last page in the page number
%
% NOTE: comment the +LP line and uncomment the -LP line to have page
%       numbers without the ``of ##'' last page reference)
%
% NOTE: uncomment the \pagestyle{empty} line to get rid of all page
%       numbers (make sure includefoot is commented out above)
%
\usepackage{fancyhdr,lastpage}
\pagestyle{fancy}
%\pagestyle{empty}      % Uncomment this to get rid of page numbers
\fancyhf{}\renewcommand{\headrulewidth}{0pt}
\fancyfootoffset{\marginparsep+\marginparwidth}
\newlength{\footpageshift}
\setlength{\footpageshift}
          {0.5\textwidth+0.5\marginparsep+0.5\marginparwidth-2in}
\lfoot{\hspace{\footpageshift}%
       \parbox{4in}{\, \hfill %
                    \arabic{page} of \protect\pageref*{LastPage} % +LP
%                    \arabic{page}                               % -LP
                    \hfill \,}}

% Finally, give us PDF bookmarks
\usepackage{color,hyperref}
\definecolor{myGreen}{rgb}{0.0,0.35,0.15}
\hypersetup{colorlinks,breaklinks,
            linkcolor=myGreen,
            urlcolor=myGreen,
            anchorcolor=myGreen,
            citecolor=myGreen}

%%%%%%%%%%%%%%%%%%%%%%%% End Document Setup %%%%%%%%%%%%%%%%%%%%%%%%%%%%


%%%%%%%%%%%%%%%%%%%%%%%%%%% Helper Commands %%%%%%%%%%%%%%%%%%%%%%%%%%%%

%%% HEADING AT TOP OF CURRICULUM VITAE

% The title (name) with a horizontal rule under it
% (optional argument typesets an object right-justified across from name
%  as well)
%
% Usage: \makeheading{name}
%        OR
%        \makeheading[right_object]{name}
%
% Place at top of document. It should be the first thing.
% If ``right_object'' is provided in the square-braced optional
% argument, it will be right justified on the same line as ``name'' at
% the top of the CV. For example:
%
%       \makeheading[\emph{Curriculum vitae}]{Your Name}
%
% will put an emphasized ``Curriculum vitae'' at the top of the document
% as a title. Likewise, a picture could be included:
%
%   \makeheading[\includegraphics[height=1.5in]{my_picutre}]{Your Name}
%
% the picture will be flush right across from the name.
\newcommand{\makeheading}[2][]%
        {\hspace*{-\marginparsep minus \marginparwidth}%
         \begin{minipage}[t]{\textwidth+\marginparwidth+\marginparsep}%
             {\large \bfseries #2 \hfill #1}\\[-0.15\baselineskip]%
                 \rule{\columnwidth}{1pt}%
         \end{minipage}}

%%% SECTION HEADINGS

% The section headings. Flush left in small caps down pseudo-margin.
%
% Usage: \section{section name}
\renewcommand{\section}[1]{\pagebreak[3]%
    \vspace{1.3\baselineskip}%
    \phantomsection\addcontentsline{toc}{section}{#1}%
    \noindent\llap{\scshape\smash{\parbox[t]{\marginparwidth}{\hyphenpenalty=10000\raggedright #1}}}%
    \vspace{-\baselineskip}\par}

%%% LISTS

% This macro alters a list by removing some of the space that follows the list
% (is used by lists below)
\newcommand*\fixendlist[1]{%
    \expandafter\let\csname preFixEndListend#1\expandafter\endcsname\csname end#1\endcsname
    \expandafter\def\csname end#1\endcsname{\csname preFixEndListend#1\endcsname\vspace{-0.6\baselineskip}}}

% These macros help ensure that items in outer-type lists do not get
% separated from the next line by a page break
% (they are used by lists below)
\let\originalItem\item
\newcommand*\fixouterlist[1]{%
    \expandafter\let\csname preFixOuterList#1\expandafter\endcsname\csname #1\endcsname
    \expandafter\def\csname #1\endcsname{\csname preFixOuterList#1\endcsname\let\oldItem\item\def\item{\pagebreak[2]\oldItem}}
    \expandafter\let\csname preFixOuterListend#1\expandafter\endcsname\csname end#1\endcsname
    \expandafter\def\csname end#1\endcsname{\let\item\oldItem\csname preFixOuterListend#1\endcsname}}
\newcommand*\fixinnerlist[1]{%
    \expandafter\let\csname preFixInnerList#1\expandafter\endcsname\csname #1\endcsname
    \expandafter\def\csname #1\endcsname{\let\oldItem\item\let\item\originalItem\csname preFixInnerList#1\endcsname}
    \expandafter\let\csname preFixInnerListend#1\expandafter\endcsname\csname end#1\endcsname
    \expandafter\def\csname end#1\endcsname{\csname preFixInnerListend#1\endcsname\let\item\oldItem}}

% An itemize-style list with lots of space between items
%
% Usage:
%   \begin{outerlist}
%       \item ...    % (or \item[] for no bullet)
%   \end{outerlist}
\newlist{outerlist}{itemize}{3}
    \setlist[outerlist]{label=\enskip\textbullet,leftmargin=*}
    \fixendlist{outerlist}
    \fixouterlist{outerlist}

% An environment IDENTICAL to outerlist that has better pre-list spacing
% when used as the first thing in a \section
%
% Usage:
%   \begin{lonelist}
%       \item ...    % (or \item[] for no bullet)
%   \end{lonelist}
\newlist{lonelist}{itemize}{3}
    \setlist[lonelist]{label=\enskip\textbullet,leftmargin=*,partopsep=0pt,topsep=0pt}
    \fixendlist{lonelist}
    \fixouterlist{lonelist}

% An itemize-style list with little space between items
%
% Usage:
%   \begin{innerlist}
%       \item ...    % (or \item[] for no bullet)
%   \end{innerlist}
\newlist{innerlist}{itemize}{3}
    \setlist[innerlist]{label=\enskip\textbullet,leftmargin=*,parsep=0pt,itemsep=0pt,topsep=0pt,partopsep=0pt}
    \fixinnerlist{innerlist}

% An environment IDENTICAL to innerlist that has better pre-list spacing
% when used as the first thing in a \section
%
% Usage:
%   \begin{loneinnerlist}
%       \item ...    % (or \item[] for no bullet)
%   \end{loneinnerlist}
\newlist{loneinnerlist}{itemize}{3}
    \setlist[loneinnerlist]{label=\enskip\textbullet,leftmargin=*,parsep=0pt,itemsep=0pt,topsep=0pt,partopsep=0pt}
    \fixendlist{loneinnerlist}
    \fixinnerlist{loneinnerlist}

%%% EXTRA SPACE

% To add some paragraph space between lines.
% This also tells LaTeX to preferably break a page on one of these gaps
% if there is a needed pagebreak nearby.
\newcommand{\blankline}{\quad\pagebreak[3]}
\newcommand{\halfblankline}{\quad\vspace{-0.5\baselineskip}\pagebreak[3]}

%%% FORMATTING MACROS

% Uses hyperref to link DOI
\newcommand\doilink[1]{\href{http://dx.doi.org/#1}{#1}}
\newcommand\doi[1]{doi:\doilink{#1}}

% For \url{SOME_URL}, links SOME_URL to the url SOME_URL
\providecommand*\url[1]{\href{#1}{#1}}
% Same as above, but pretty-prints SOME_URL in teletype fixed-width font
\renewcommand*\url[1]{\href{#1}{\texttt{#1}}}

% For \email{ADDRESS}, links ADDRESS to the url mailto:ADDRESS
\providecommand*\email[1]{\href{mailto:#1}{#1}}
% Same as above, but pretty-prints ADDRESS in teletype fixed-width font
%\renewcommand*\email[1]{\href{mailto:#1}{\texttt{#1}}}

%\providecommand\BibTeX{{\rm B\kern-.05em{\sc i\kern-.025em b}\kern-.08em
%    T\kern-.1667em\lower.7ex\hbox{E}\kern-.125emX}}
%\providecommand\BibTeX{{\rm B\kern-.05em{\sc i\kern-.025em b}\kern-.08em
%    \TeX}}
\providecommand\BibTeX{{B\kern-.05em{\sc i\kern-.025em b}\kern-.08em
    \TeX}}

% Custom hyphenation rules for words that LaTeX has trouble with
\hyphenation{bio-mim-ic-ry bio-in-spi-ra-tion re-us-a-ble pro-vid-er}

%%%%%%%%%%%%%%%%%%%%%%%% End Helper Commands %%%%%%%%%%%%%%%%%%%%%%%%%%%

%%%%%%%%%%%%%%%%%%%%%%%%% Begin CV Document %%%%%%%%%%%%%%%%%%%%%%%%%%%%

\begin{document}
\makeheading{Philippe Desjardins-Proulx}

\section{Contact}

Graduate (Ph.D.) student\\
\href{http://www.bio.uqam.ca/}{Department of Biology, Université du Québec à Montréal}\\
\href{http://qcbs.ca/}{Quebec Center for Biodiversity Science, McGill University, Canada.}\\
\textit{Phone:} +1-418-732-9877 \\
\textit{E-mail:} \email{philippe.d.proulx@gmail.com}\\
\textit{WWW:} \href{http://phdp.github.com/}{http://phdp.github.com/}\\
\textit{GitHub:} \href{https://github.com/PhDP}{https://github.com/PhDP}
%\textit{BitBucket: } \href{https://bitbucket.org/PhDP}{https://bitbucket.org/PhDP}

\section{Citizenship}

Canada.

\section{Languages}

French \& English. Basic knowledge of Japanese.

\section{Professional Interests}

\textbf{Artificial Intelligence; Machine Learning; Artificial Neural Networks;
Complexity; Bayesian Inference}; Information theory; Biodiversity; Theoretical
Population Genetics; Theoretical Ecosystem Ecology.

\section{Other \\Interests}

\textbf{Effective Technical Writing in English}; Epistemology; Scientific Computing;
Functional Programming; Open Source Software; Computational Finance.

\section{Major Awards}

\href{http://www.nserc-crsng.gc.ca/}{\textbf{From:} Natural Sciences and Engineering Research Council of Canada}
\begin{innerlist}
  \item \href{http://www.nserc-crsng.gc.ca/students-etudiants/pg-cs/bellandpostgrad-belletsuperieures_eng.asp}{\textbf{Award:} Alexander Graham Bell Graduate Scholarship}, 09/2012--08/2015
  \item \textbf{Value:} 105 000\$ ($\approx$ 105 000 USD $\approx$ 8 150 000 JPY, 2012 est.)
\end{innerlist}

\section{Education}

\href{http://www.bio.uqam.ca/}{\textbf{Department of Biology, Université du Québec à Montréal, Montréal, Canada.}}
\begin{outerlist}
\item[] Ph.D., September 2012 -- August 2015 \emph{[expected]}
        \begin{innerlist}
        %\item Thesis Topic: \emph{Design and Analysis of Optimal Task-Processing Agents}
        \item Thesis Proposal: \emph{Artificial Intelligence and the Puzzle of Genomic Diversity}
        %\item Candidacy: \emph{Research Problems in Distributed Control for Energy Systems}
        \item Adviser:
              \href{http://chaire-eec.uqar.ca/dom.php}
                   {Dr. Dominique Gravel}
        \item Area of Study: Artificial Intelligence (machine learning: neural networks) and its applications to biodiversity.
        \end{innerlist}
\end{outerlist}

\hspace{.5cm}

\href{http://engineering.uic.edu/}{\textbf{College of Engineering, University of Illinois at Chicago, Chicago, USA.}}
\begin{outerlist}
\item[] Graduate Certificate in Bioinformatics, 2012,
        \begin{innerlist}
          \item Area of Study: Machine learning (Artificial Intelligence) and biostatistics.
        \end{innerlist}
\end{outerlist}

\hspace{.5cm}

\href{http://www.uquebec.ca/}{\textbf{Université du Québec, Québec, Canada.}}
\begin{outerlist}
\item[] B.S., 2009,
        \begin{innerlist}
          \item Major in Biology,
          \item Minor in Mathematics \& Computer Science.
        \end{innerlist}
\end{outerlist}

\section{Refereed Journal Publications}

\begin{bibenum}

  \item \textbf{P Desjardins-Proulx}, EP White, JJ Adamson, K Ram, T Poisot, and
D Gravel. Developing a preprint culture in biology.\\
\emph{Submitted to PLOS Biology (in review)}

  \item R Vergilino, TA Elliott, \textbf{P Desjardins-Proulx}, TJ Crease and F
Dufresne.  Evolution of a transposon in \emph{Daphnia} hybrid genomes.\\
\emph{Accepted in Mobile DNA}

  \item D Ai, \textbf{P Desjardins-Proulx}, C Chu, and G Wang. The influence of
immigration and dispersal limitation on the repeatability of niche and neutral
communities. \emph{PLOS ONE} 7(9): e46164, 2012.\\ DOI:
10.1371/journal.pone.0046164

  \item \textbf{P Desjardins-Proulx} and D Gravel. A complex speciation-richness
relationship in a simple neutral model. \emph{Ecology and Evolution} 2(8):
1781--1790, 2012.\\ DOI: 10.1002/ece3.292

  \item \textbf{P Desjardins-Proulx} and D Gravel. How likely is speciation in
neutral ecology? \emph{The American Naturalist} 179(1):137-144, 2012.\\ DOI:
10.1086/663196\\

\end{bibenum}

\section{Other Contributions}

\begin{bibenum}
    \item \textbf{P Desjardins-Proulx}. The case for arXiv and a broader
conception of peer-reviews. Invited blog, International Network of
Next-Generation Ecologists, 2012.\\
    \href{http://www.innge.net/?q=node/330}{http://www.innge.net/?q=node/330}.

    \item \textbf{P Desjardins-Proulx}. A foot in the neutral trap.\\
    Invited comment for \emph{Trends in Ecology \& Evolution}, 2012.

    \item \textbf{P Desjardins-Proulx}. L'origine de la Biodiversit\'e. Le
Mouton Noir, Mai-Juin. Cahier Sp\'ecial sur la Biodiversit\'e p.2, 2010.
\textit{Selected and republished by GaiaPresse, a group sponsored by the
Universit\'e Laval}.
\end{bibenum}

\section{Teaching \&\\Training}

\href{http://www.uquebec.ca/}{\textbf{Université du Québec, Québec, Canada.}}
\begin{innerlist}
  \item 2013. I organized group studies of MacKay's \emph{Information Theory, Inference, and Learning Algorithms}.
  \item 2012. CUDA training (intensive one-day course).
  \item 2012. Scientific computing with C (grad. students/post-docs).
  \item 2011. Scientific computing with C (grad. students/post-docs).
\end{innerlist}

\section{Referee \\Service}

\emph{Journal of Theoretical Biology, Theoretical Ecology, Acta Biotheoretica, Journal of Plant Ecology}.

\section{Computer \\Skills}

\begin{innerlist}
  \item \textbf{Programming languages:}
  \begin{innerlist}
    \item[--] \textbf{Advanced:}            C$+$$+$11, C.
    \item[--] \textbf{Intermediate:}        JavaScript, Haskell, Java, Go, Python, R.
    \item[--] \textbf{Basic:}               Ruby, MatLab/Octave.
  \end{innerlist}
  \item \textbf{Tools \& Frameworks}
  \begin{innerlist}
    \item[--] \textbf{Operating systems:}   Linux (Debian/Ubuntu, Arch).
    \item[--] \textbf{Writing:}             \LaTeXe, LibreOffice.
    \item[--] \textbf{Compilers:}           llvm/clang, gcc, intel, ghc.
    \item[--] \textbf{Parallel computing:}  CUDA, OpenMP.
    \item[--] \textbf{Linear algebra:}      Armadillo.
    \item[--] \textbf{Revision control:}    git, mercurial.
    \item[--] \textbf{Database:}            Redis, MongoDB.
    \item[--] \textbf{Web:}                 Node.js (Express).
  \end{innerlist}
  \item \textbf{Web sites:}
  \begin{innerlist}
    \item[--] Personal page: \href{http://phdp.github.com}{http://phdp.github.com}.
    \item[--] TEE's website: \href{http://chaire-eec.uqar.ca}{http://chaire-eec.uqar.ca}.
  \end{innerlist}
  \item \textbf{Primary working tools:}
  \begin{innerlist}
    \item[--] \LaTeXe, vim, tmux.
  \end{innerlist}
\end{innerlist}

\section{Professional \\Memberships}

\begin{innerlist}
  \item Institute of Electrical and Electronics Engineers \hfill 2012--...
  \item International Society for Computational Biology \hfill 2010--...
  \item Society for the Study of Evolution \hfill 2008--...
%  \item Society for Molecular Biology and Evolution \hfill 2009--...
%  \item The American Society of Naturalists \hfill 2012--...
  \item Quebec Center for Biodiversity Science \hfill 2012--...
\end{innerlist}

\section{Graduate \\Courses}

\begin{innerlist}
  \item 2012. Datamining (machine learning) [A (4.0/4.0), 4 credits] \hfill \href{http://www.uic.edu/}{UIC}
  \item 2011. Biostatistics [A (4.0/4.0), 4 credits] \hfill \href{http://www.uic.edu/}{UIC}
  \item 2010. Intro. to bioinformatics [A (4.0/4.0), 4 credits] \hfill \href{http://www.uic.edu/}{UIC}
  \item 2010. Reading course on Ancestral Recombination Graphs [A, 3 credits] \hfill \href{http://www.uqar.ca/}{UQAR}
\end{innerlist}

\section{Referees}

\textbf{Dr.~Dominique Gravel}
\begin{innerlist}
  \item Professor (Universit\'e du Qu\'ebec \`a Rimouski)
  \item Canada Research Chair.
  \item e-mail:~\href{MAILTO:dominique_gravel@uqar.qc.ca}{dominique\_gravel@uqar.qc.ca}
  \item phone: 1.418.723.1986 \#1752
  \item[$\star$] \emph{I worked as a research profesionnal in Dr. Gravel's lab
    from September 2009 to August 2012, we also collaborated on many scientific
    projects. I am a Ph.D. student in his lab since September 2012.}
\end{innerlist}

\halfblankline

\textbf{Dr.~James Rosindell}
\begin{innerlist}
  \item Post-doctoral researcher, Imperial College London, UK.
  \item e-mail:~\href{MAILTO:j.rosindell@imperial.ac.uk}{j.rosindell@imperial.ac.uk}
  \item phone: +44 (0)2075 942263
  \item[$\star$] \emph{I have collaborated with Dr. Rosindell on several occasions.}
\end{innerlist}

\end{document}

